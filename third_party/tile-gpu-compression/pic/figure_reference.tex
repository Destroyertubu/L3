% LaTeX figure reference for nvcomp vs G-LeCo comparison
%
% Usage: Copy this code into your Overleaf document where you want to insert the figure
% Make sure the PDF file is uploaded to your Overleaf project in the same relative path

\begin{figure}[htbp]
    \centering
    \includegraphics[width=\textwidth]{pic/nvcomp_L3_comparison.pdf}
    \caption{Performance comparison between nvcomp algorithms and G-LeCo on six datasets.
    \textbf{Left:} Compression ratio for each dataset (lower is better).
    \textbf{Right:} Decompression throughput comparison. G-LeCo achieves 8.88$\times$ average
    compression ratio and 994.84 GB/s average throughput, outperforming all nvcomp algorithms.}
    \label{fig:nvcomp_L3_comparison}
\end{figure}

% To reference this figure in text, use:
% As shown in Figure~\ref{fig:nvcomp_L3_comparison}, G-LeCo demonstrates...

% Alternative caption (shorter version):
% \caption{Performance comparison between nvcomp algorithms and G-LeCo on compression ratio
% (left, 6 datasets) and decompression throughput (right).}

% If you want to adjust the width, you can use:
% \includegraphics[width=0.9\textwidth]{pic/nvcomp_L3_comparison.pdf}
% or
% \includegraphics[width=\linewidth]{pic/nvcomp_L3_comparison.pdf}

% For two-column format papers, you might want to use:
% \begin{figure*}[htbp]
%     \centering
%     \includegraphics[width=\textwidth]{pic/nvcomp_L3_comparison.pdf}
%     \caption{...}
%     \label{fig:nvcomp_L3_comparison}
% \end{figure*}

% ============================================================================
% Figure 2: Tile methods vs G-LeCo comparison
% ============================================================================

\begin{figure}[htbp]
    \centering
    \includegraphics[width=\textwidth]{pic/side_by_side_performance_percentage.pdf}
    \caption{Performance comparison of Tile-based methods and G-LeCo on six SOSD datasets.
    \textbf{Left:} Compressed size percentage (lower is better).
    \textbf{Right:} Decompression throughput. G-LeCo achieves the best compression ratio
    (8.88$\times$ average) while maintaining competitive throughput (994.84 GB/s average).}
    \label{fig:tile_L3_comparison}
\end{figure}

% To reference this figure in text, use:
% As shown in Figure~\ref{fig:tile_L3_comparison}, G-LeCo demonstrates...

% Alternative caption (shorter version):
% \caption{Performance comparison of Tile-FOR, Tile-DFOR, Tile-RFOR, and G-LeCo on
% compression ratio (left) and throughput (right) across six SOSD datasets.}

% For two-column format papers:
% \begin{figure*}[htbp]
%     \centering
%     \includegraphics[width=\textwidth]{pic/side_by_side_performance_percentage.pdf}
%     \caption{...}
%     \label{fig:tile_L3_comparison}
% \end{figure*}
